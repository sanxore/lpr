\chapter{Exigences de sécurité pour les applications e-santé mobiles}

Ce chapitre est consacré aux exigences de sécurité pour les systèmes e-santé, en particulier les applications mobiles. Il est organisé en deux parties: La première présente l'e-santé mobile en montrant la contribution de la mobilité dans l'amélioration du système de soins ainsi le défi de sécurité à relever. La deuxième partie indique les politiques de privacy et de sécurité à mettre en place par les fabricants des applications e-santé mobiles afin de protéger les renseignements personnels.

\section{Applications e-santé mobiles}

Smartphones, tablettes, micro-PC et services Internet mobile à haut débit qui les accompagnent ont envahi notre quotidien ces dernières années. Il est désormais possible et facile pour quiconque de consulter ses mails, lire la presse en ligne ou même commander son billet de train sur Internet depuis n’importe où. Ce développement technologique intéresse évidemment la e-santé, où l’accès aux données est primordial $[W5]$.

\vspace{6pt}
\paragraphmark

\subsection{Pourquoi la mobilité ?}

L’internet mobile, c’est-à-dire la possibilité d’accéder au world wide web sans être connecté à une prise téléphonique, a rendu l’accès à l’information tellement facile qu’il est désormais omniprésent dans nos vies. Conjointement au développement du réseau, les appareils permettant de se connecter sont devenus meilleur marché, plus faciles à utiliser et plus autonomes. Disponible quasiment partout, l’internet s’est en quelques années totalement intégré à notre vie quotidienne, parce qu’il nous rend de nombreux services : la réservation rapide de places de cinéma, la consultation d’horaires de transport ou la recherche d’une pharmacie de garde est désormais possible partout du bout des doigts $[W5]$.

\vspace{6pt}
\paragraphmark

On imagine alors assez vite les applications de ces technologies à la santé : un médecin pourrait observer les radios d’un patient reçues sur son smartphone pendant une consultation à domicile, ou visualiser aisément les antécédents médicaux d’un accidenté de la route, directement sur le lieu de l’accident $[W5]$.

\subsection{Le défi de la sécurité}

Tout d'abord, la sécurité des données de santé sensibles ainsi consultées et échangées est essentielle. En effet, il est d’autant plus nécessaire de garantir la confidentialité des données quand celles-ci peuvent être portées sur des dispositifs mobiles qui sont plus facilement perdus... Aussi il est impératif que les logiciels utilisés prévoient des possibilités de masquage et de cryptage des données.
Le bon développement des systèmes mobiles passe donc par la garantie d’un système sûr $[W5]$. C’est pourquoi il est nécessaire de définir des politiques de sécurité et de confidentialité que les industriels de l’informatique doivent respecter dans leurs logiciels.

\section{Politiques de privacy et de sécurité}

Les politiques sont destinées à définir les contours d’un « espace de confiance » indispensable au déploiement des applications e-santé mobiles. On distingue deux types: Les politiques de privacy (de confidentialité) et les politiques de sécurité.

\subsection{Politique de privacy}

\subsubsection{Définition}

Une politique de confidentialité \footnote{La confidentialité a été définie par l'Organisation internationale de normalisation (ISO) comme « le fait de s'assurer que l'information n'est seulement accessible qu'à ceux dont l'accès est autorisé », et est une des pierres angulaires de la sécurité de l'information. $[W7]$} est un contrat qui décrit comment une société retient, traite, publie et efface les données transmises par ses clients. Par exemple, un site web qui exige une inscription pour participer activement à ses forums devrait offrir une telle politique pour les données à caractères personnels qui lui sont confiées : âge, sexe, niveau d'études, etc $[W6]$.

\vspace{6pt}
\paragraphmark

Une politique de confidentialité devrait habituellement contenir des clauses qui décrivent comment les informations personnelles sont archivées, comment elles peuvent être utilisées, les personnes à qui elles pourraient être transmises, les mesures de protection mises en place. Dans le cas de sites web, elle devrait également indiquer si le site a recours à des cookie $[W6]$.

\subsubsection{Politique de privacy pour les applications m-santé}

En conformité avec la loi de la protection des renseignements personnels, aucune des informations (nom, prénom, âge, sexe, etc) transmises à ou recueillies par l'application e-santé mobile ne doit, en aucune circonstance, être cédée, louée ou vendue à une tierce partie, qu’elle soit publique ou privée.

\subsection{Politique de sécurité}

\subsubsection{Définition}

Une politique de sécurité est un plan d'actions définies pour maintenir un certain niveau de sécurité. Elle reflète la vision stratégique de la direction de l'organisme (PME, PMI, industrie, administration, état, unions d'états …) en matière de sécurité informatique $[W8]$.

\vspace{6pt}
\paragraphmark

Elle définit les objectifs de sécurité des systèmes informatiques d'une organisation. La définition peut être formelle ou informelle. Les politiques de sécurité sont mises en vigueur par des procédures techniques ou organisationnelles. Une mise en œuvre technique définit si un système informatique est sûr ou non sûr $[W8]$.

\vspace{6pt}
\paragraphmark

Pour définir une politique de sécurité, il faut suivre les quatre étapes suivantes $[W8]$:

\vspace{6pt}
\paragraphmark

\begin{itemize}
	\item Elaborer des règles et des procédures, installer des outils techniques dans les différents services de l'organisation (autour de l'informatique);
	\item Définir les actions à entreprendre et les personnes à contacter en cas de détection d'une intrusion;
	\item Sensibiliser les utilisateurs aux problèmes liés à la sécurité des systèmes;
	\item Préciser les rôles et responsabilités.
\end{itemize}

\subsubsection{Politique de sécurité pour les applications m-santé}

Les appareils mobiles ne se contrôlent pas comme des PC de bureau ou des PC portables. Lorsqu'il s'agit d'administrer des smartphones, un pare-feu devient quasiment inefficace. Il faut donc intégrer la sécurité à toutes les applications personnalisées sans exception, mettre en œuvre une administration centrale des services en cloud, contrôler le mode de mise à disposition des applications afin d'assurer que les appareils sont conformes aux politiques de sécurité avant qu'ils n'accèdent aux données. Ainsi il faut considérer chacune des plates-formes mobiles (Apple iOS, Google Android, BlackBerry et Microsoft Windows Phone.) prises en charge et aborder les applications en cloud comme autant de plates-formes distinctes. Toutes ces ressources font certes appel à des certificats et à des réseaux cryptés, mais l'essentiel n’en reste pas moins de s'assurer que les utilisateurs exploitent les applications et les données de manière sécurisée $[W9]$.