\chapter{Analyse et spécification}

Ce chapitre est consacré aux exigences de sécurité pour les systèmes e-santé, en particulier les applications mobiles. Il est organisé en deux parties: La première présente l'e-santé mobile en montrant la contribution de la mobilité dans l'amélioration du système de soins ainsi le défi de sécurité à relever. La deuxième partie indique les politiques de privacy et de sécurité à mettre en place par les fabricants des applications e-santé mobiles afin de protéger les renseignements personnels.
\cleardoublepage

\section{Analyse de l’existant}
\section{Revue de littérature}
\subsection{Qu'est ce que le Data Mining ?}
La fouille de données, français pour le "Data Mining" est le processus de recherche et de découverte d'auparavant inconnus et potentiellement intéressants modèles dans les grands ensembles de données \cite{def-DM}. L'information 'minée' est typiquement représentée par un modèle de la structure sémantique de l'ensemble de données, où le modèle peut être utilisé sur de nouvelles données à des fins de prédiction ou de classification. Alternativement, des experts humains du métier en question peuvent choisir d'examiner manuellement le modèle, à la recherche d'éléments qui expliqueraient des caractéristiques précédemment mal comprises ou inconnues du domaine d'étude.
\paragraph{}
Brynjolfsson, Hitt, et Kim\cite{data-driven-des} ont mené des recherches empiriques et ont conclu que la performance des organisations est directement liée à leur capacité à prendre des décisions dirigées par les données. Il est donc important de comprendre les facteurs de succès nécessaires pour adopter des techniques de fouilles de données dans les organisations.
\subsection{Les études de prévisions en matière de fertilisants}
\subsection{Prévision à court-terme de la demande}
\section{Compréhension du problème}
\section{Spécification}
	\subsection{Spécification fonctionelle}
	\subsection{Spécification technique}