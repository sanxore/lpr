\chapter{Collecte, compréhension et préparation des données.}
\epigraph{“Data! Data! Data!” he cried impatiently. “I can’t make bricks without clay.”}{Sherlock Holmes}	
\cleardoublepage
\newcommand{\reels}{\mathbb{R}}
	\section{Compréhension et préparation des données locales\protect\footnote{Données disponibles au sein du portail Business Intelligence de l'OCP}}
	\subsection{Compréhension des données locales}
	Comme rappelé dans la section \ref{CGP}. Les travaux de nos prédécesseurs dans leurs efforts de moderniser le portail \textit{Business Intelligence} de l'OCP se sont arrêtés à automatiser l'archivage des données se rapportant aux historiques de ventes en terme de prix et volumes en un premier lieu\cite{CHEMLAL} avant de concevoir le socle OLAP\footnote{le traitement analytique en ligne (OnLine Analytical Processing, OLAP)\nomenclature{\textbf{OLAP : }}{OnLine Analytical Processing} est un type d'application informatique orienté vers l'analyse sur-le-champ d'informations selon plusieurs axes, dans le but d'obtenir des rapports de synthèse} dans la vue de générer des rapports synthétiques concernant les historiques des échanges du marché des phosphates ensuite\cite{NACER}.\\
	Seules sont ainsi présentes les données concernant les échanges internationaux en terme de produits phosphatés et ceux-ci sont présents sous deux différentes formes de données.
					\begin{wrapfigure}[9]{r}{5 cm}
						\raggedleft
						\fbox{\includegraphics[scale=0.5]{IFA-txt}}
						\raggedleft
						\caption{Lecture "machine" du .pdf de la "Trade Matrix" de la figure \ref{fig:IFA-PDF}}
						\label{fig:IFA-TXT}
					\end{wrapfigure}
	La première est notre format de fichier source: des fichiers .pdf  non structurés vis-à-vis de notre besoin (figure \ref{fig:IFA-PDF}), et qui présentent:
		\begin{itemize}
		\item L'avantage d’être à jour, exhaustifs et dont la véracité est certifiée par un organisme international (IFA\nomenclature{\textbf{IFA : }}{International Fertilizer industry Association})
		\item L’inconvénient d’être flexible pour la lecture humaine mais ne présentant pas une grammaire machine formelle rendant possible une analyse syntaxique, comme en témoigne la figure \ref{fig:IFA-TXT}.
			\begin{figure}[H]
			    		\raggedright
		    			\fbox{\includegraphics[scale=0.3]{IFA_EX}}
		    			\captionsetup{justification=raggedright,
		    			singlelinecheck=false
		    			}
			    		\caption{Exemple d'une "Trade Matrix"\protect\footnote{Matrice des imports/exports entre les pays du monde, deux-à-deux, des phosphates et produits dérivés} dans le rapport\\trimestriel de l'IFA}
			    		\label{fig:IFA-PDF}
			\end{figure}	
		\end{itemize}
		
		\paragraph{}
		La seconde est structurée dans des datamarts dont une sortie de requête est présentée dans la figure \ref{fig:DMOCP}.
	Ceci est notre format de données cible 
	et présente:
	\begin{itemize}
	\item L'avantage d’offrir le maximum de flexibilité pour le requêtage, et d'être de grande qualité en terme de disponibilité et de véracité, puisque celui-ci a été soigneusement introduit à la main\footnote{À travers une lecture "humaine" des .pdf présentés par la figure \ref{fig:IFA-PDF}}.
	\item L’inconvénient d'être prohibitif en temps et en ressources humaines.
	 En effet, nous avons constaté un retard datant de fin décembre 2013 par rapport aux derniers .pdf reçus par l'OCP.
	\end{itemize}
	\begin{figure}[H]
		    		\centering
	    			\includegraphics[scale=0.325]{Table}
		    		\caption{Exemple d'une sortie de requête sur le Datamart OCP-CM}
		    		\label{fig:DMOCP}
	\end{figure}
	\subsection{Consolidation des données locales}
	Il est naturel d'adresser les problèmes de disponibilité de données aux premiers abords. La consolidation désigne la collection et l'intégration des données de sources multiples en une unique destination. Durant ce processus, nous unifierons les deux types de formats de données. Ceci nous permettra de présenter les données de manière plus flexible, tout en facilitant leur analyse effective. Ceci nous amène à considérer le format adopté par le datamart OCP-CM comme format cible de consolidation et les fichiers .pdf comme format source.
	\subsubsection{Conception de la solution}
	\paragraph{Spécification fonctionelle:\\}
	L'OCP reçoit régulièrement des rapports trimestriels présentant la situation du marché accompagnée des mouvements observés des produits fertilisants. Ces mouvements sont reportés sur des tableaux tel que celui présenté dans la figure \ref{fig:IFA-PDF}. Notre solution doit ainsi considérer les nuances fonctionnelles suivantes :
	\begin{enumerate}
	\item Les rapports diffèrent par leur granularité. En effet ceux-ci peuvent être:
		\begin{itemize}
		\item DET : Détaillés. Présentant l'historique des mouvements de produits entre les pays du monde deux-à deux.
		\begin{figure}[h]
					    		\centering
					    		\fbox{\includegraphics[scale=0.35]{det}}
					    		\caption{Exemple d'en-tête des tables du format DET des rapports IFA.}
				\end{figure}
		\item AGG : Agrégés. Présentant l'historique des mouvements entre les pays agrégés selon la région du monde à laquelle ceux-ci appartiennent.
		\begin{figure}[h]
			    		\centering
			    		\fbox{\includegraphics[scale=0.35]{agg}}
			    		\caption{Exemple d'en-tête des tables du format  AGG des rapports IFA.}
		\end{figure}
		\end{itemize}
	\item Les rapports diffèrent par les normes des chiffres rapportés. En effet ceux-ci peuvent être:
		\begin{itemize}
		\item NOT CUMULATED : Les chiffres rapportés pour le trimestre ${Q_i}$ sont bruts et représentent uniquement les ventes ayant effectivement eu lieu durant ce trimestre.
		\item CUMULATED : Les chiffres rapportés pour le trimestre ${Q_i}$ sont cumulés, i.e ${Q_i = \sum_{j=1}^{j=i} Q_j}$
		\item ANN : Les chiffres rapportés représentent toutes les ventes de l'année, i.e ${Q_i = \sum_{j=1}^{j=4} Q_j}$
		\end{itemize}
	\item Les attributs des enregistrements du datamart OCP-CM, sont des champs obligatoirement \textbf{NOT NULL} et sont les suivants:
	\begin{itemize}
	\item "Importing.countries" : Pays de destination de l'enregistrement-vente.
	\item "Region..IFA." : Région à laquelle appartient le pays de destination selon le découpage IFA.
	\item "Region..OCP." : Région à laquelle appartient le pays de destination selon le découpage OCP.
	\item "Exporting.countries" : Pays d'origine de l'enregistrement-vente.
	\item "Product" : Produit de l'enregistrement-vente. (MAP\footnote{Monoammonium phosphate},DAP\footnote{Diammonium phosphate},PA\footnote{Acide phosphorique},TSP\footnote{Triple superphosphate},ROCK\footnote{Minerai du phosphate brut en roche.})
	\item "Year" : Année de l'enregistrement-vente.
	\item "Quarter" Trimestre de l'enregistrement-vente.
	\item "Code" : Code de Synthèse des champs précédents.
	\item "kT" :  Poids de l'enregistrement-vente en kT\footnote{Kilotonne = $10^6$ kilogramme.} de produit.           
	\item "P2O5.Product" : Poids équivalent du produit en P2O5\footnote{Pentoxyde de phosphore, la molécule de base des engrais phosphatés.}
	\item "Cumulated.Not.cumulated" : Drapeau indiquant la norme trimestrielle de l'enregistrement-vente.
	\item "AGG.DET.ANN" : Drapeau indiquant la granularité de l'enregistrement-vente
	\end{itemize}
		Ainsi au-delà de la lecture des données contenues au sein des documents .pdf, ceux-ci doivent être:
		\begin{itemize}
		\item \textbf{Décumulés:} Des enregistrement des volumes unitaires par trimestre doivent être créés.
		\item \textbf{Convertis en P2O5:} Les données contenues dans les .pdf représentent les volumes échangés par kT qu'ils faut convertir selon la concentration du produit de l'enregistrement-vente en P2O5.
		\item \textbf{Normés:} La nature de l'agrégation trimestrielle de l'enregistrement-vente doit être spécifiée.
		\end{itemize}
	\end{enumerate}
	\paragraph{Spécification technique:\\}
	
		\begin{figure}[h]
		    		\centering
		    		\includegraphics[scale=0.6]{dataflow}
		    		\caption{Flux de données de consolidation.}
		    		\label{fig:DF}
		\end{figure}
	\subsubsection{Système de fichiers de la solution}
	\subsubsection{Structuration de la donnée}
	\subsection{Audit et description des données locales consolidées}
	\section{Collecte et préparation des données externes}
	L'étendue des données disponibles au sein du système d'information de l'OCP ne pouvant donner qu'une vue réduite de la situation du marché et ne peuvent de par leur définition révéler les structures socio-économiques,  et de politiques agraires sous-jacentes aux demandes en produits phosphatés, nous nous attellerons à la tâche d'enrichir cette base de données historiques par des données quantitatives concernant les pays du monde.
	\par
	Nous décrivons dans ce qui suit ce processus. 
	\subsection{Collecte des données externes}
	\subsubsection{Énumération et définitions des variables souhaitées }\label{exoList}
			Guidés par nos lectures bibliographiques résumées dans les sections \ref{read1} et \ref{read2}, nous nous fixons l'objectif de récolter les indicateurs de politiques agraires et socioéconomiques suivants:
	\begin{itemize}
		\item \textbf{ Access to electricity, rural (\% of rural population):} Fraction de paysans ayant accès à l'électricité.
		\item \textbf{ Access to non-solid fuel, rural (\% of rural population):} Fraction de paysans ayant accès aux fuels non solides.
		\item \textbf{ Account at a financial institution (\% age 15+):} Fraction de personnes âgées de +15 ans ayant un compte chez une institution bancaire.
		\item \textbf{ Net enrollment rate, primary (\% of primary school age children):} Fraction d'enfants scolarisés.
		\item \textbf{ Net national income per capita (constant 2005 US\$):} PIB\footnote{Produit intérieur brut, un des agrégats majeurs des comptes nationaux, il vise à quantifier — pour un pays et une année donnés — la valeur totale de la « production de richesse » effectuée par les agents économiques résidant à l’intérieur de ce territoire (ménages, entreprises, administrations publiques).} par habitant, inflation ajustée au dollar US fin 2005.
		\item \textbf{ Literacy rate, adult total (\% of people ages 15 and above):} Fraction de personnes alphabétisées âgées de +15ans.
		\item \textbf{ Agricultural irrigated land (\% of total agricultural land):} Fraction de terres irriguées parmi les terres exploitées pour l'agriculture.
		\item \textbf{ Agricultural land (\% of land area):} Fraction des terres agricoles de la surface totale du pays.
		\item \textbf{ Agricultural tractors per 100 sq. km of arable land:} Nombre de tracteurs par 100 km² de terres arables.
		\item \textbf{ Agriculture, value added (\% of GDP):} Fraction de la valeur ajoutée agricole du PIB.
		\item \textbf{ Agriculture value added per worker (constant 2005 US\$):} Valeur ajoutée agricole par ouvrier agricol, inflation ajustée au dollar US fin 2005.
		\item \textbf{ All education staff compensation, total (\% of total expenditure in public institutions):} Fraction des dépenses en éducation des dépenses en institutions publiques.
		\item \textbf{ Annual freshwater withdrawals, agriculture (\% of total freshwater withdrawal):} Fraction du volume d'eau utilsée à des fins agricoles de la totalité de l'eau consommée.
		\item \textbf{ Arable land (\% of land area):} Fraction de terres arables de la surface totale du pays.
		\item \textbf{ Arable land (hectares per person):} Nombre d'hectares de terres arables par personne.
		\item \textbf{ Birth rate, crude (per 1,000 people):} Nombre de naissances par 1000 personnes.
		\item \textbf{ Cereal yield (kg per hectare):} Rendement des céréales en kilogramme par hectare.
		\item \textbf{ Commercial bank branches (per 100,000 adults):} Nombre d'agences bancaires par 100,000 adultes.
		\item \textbf{ Consumer price index (2010 = 100):} L'indice des prix à la consommation (IPC) mesure l'évolution du niveau moyen des prix des biens et services consommés par les ménages, pondérés par leur part dans la consommation moyenne des ménages. Harmonisé pour permettre une comparaison entre les pays à fin 2010.
		\item \textbf{ Cost to export (US\$ per container):} Coût en Dollars US de l'export d'un conteneur de marchandises.
		\item \textbf{ Cost to import (US\$ per container):} Coût en Dollars US de l'import d'un conteneur de marchandises.
		\item \textbf{ Crop production index (2004-2006 = 100):} 
		L'indice de production des cultures montre la production agricole pour chaque année par rapport à la période de base de 2004 à 2006. Cet indice porte sur l'ensemble des cultures à l'exception des cultures fourragères. Les regroupements par région et par revenu des indices de production de la FAO\nomenclature{\textbf{FAO : }}{Food and Agriculture Organization of the United Nations} sont calculés à partir des valeurs sous-jacentes en dollars US et normalisés par rapport à la période de référence de 2004 à 2006.
		\item \textbf{ Droughts, floods, extreme temperatures (\% of population, average 1990-2009):} Pourcentage moyen annuel entre 1990 et 2009 de la population affectée par les catastrophes naturelles classifiées comme sécheresses, inondations et évènements climatiques extrêmes.
		\item \textbf{ Employment in agriculture (\% of total employment):} Fraction des ouvriers agricoles de l'ensemble des employés.
		\item \textbf{ Food production index (2004-2006 = 100):} L'indice de production alimentaire porte sur les cultures vivrières qui sont considérées comme comestibles et qui contiennent des nutriments et normalisées par rapport à la période de référence de 2004 à 2006.
		\item \textbf{ GDP per capita (constant 2005 US\$):} PIB par habitant. Inflation ajustée à fin 2005.
		\item \textbf{ Household final consumption expenditure (constant 2005 US\$):}  La consommation privée désigne la valeur marchande de tous les biens et services, y compris les produits durables achetés par les ménages.
		\item \textbf{ Lending interest rate (\%):} Le taux d'intérêt perçu par les banques sur les prêts accordés aux clients.	
		\item \textbf{ Life expectancy at birth, total (years):} L'espérance de vie à la naissance indique le nombre d'années qu'un nouveau-né devrait vivre si les règles générales de mortalité au moment de sa naissance devaient rester les mêmes tout au long de sa vie.
		\item \textbf{ Livestock production index (2004-2006 = 100):} L'indice de production animale comprend la production de viande et de lait de toutes sources, les produits laitiers tels que le fromage, les œufs, le miel, la soie brute, la laine ainsi que les peaux et les cuirs.
		\item \textbf{ Logistics performance index:} La note globale de l'indice de performance de la logistique reflète les perceptions relatives à la logistique d'un pays basées sur l'efficacité des processus de dédouanement, la qualité des infrastructures commerciales et des infrastructures de transports connexes, la facilité de l'organisation des expéditions à des prix concurrentiels, la qualité des services d'infrastructure, la capacité de suivi et de traçabilité des consignations et la fréquence avec laquelle les expéditions arrivent au destinataire dans les délais prévus. L'indice varie continuellement de 1 à 5 et la note la plus élevée représente la meilleure performance.
		\item \textbf{ Low-birthweight babies (\% of births):} Fraction des nouveau-nés pesant moins de 2 500 grammes des naissances totales.
		\item \textbf{ Net migration:} Nombre d'immigrants total moins le nombre d'émigrants annuel, comprenant à la fois les citoyens et les non citoyens.
		\item \textbf{ Permanent cropland (\% of land area):} Fraction des terres occupées par des cultures pour de longues périodes et qui doivent être replantées après chaque récolte de la surface totale du pays.
		\item \textbf{ Population density (people per sq. km of land area):} Densité des habitants en personne par km².
		\item \textbf{ Population growth (annual \%):} Croissance relative annuelle de la population.
		\item \textbf{ Rural population (\% of total population):} Fraction rurale de la population.
		\item \textbf{ Rural poverty gap at national poverty lines (\%):} L'écart de pauvreté par rapport au seuil national de la pauvreté en milieu rural est le manque à gagner pour remonter au-dessus du seuil de la pauvreté (en considérant que les non pauvres ont un manque à gagner de zéro) exprimé en pourcentage du seuil national de la pauvreté en milieu urbain.Cette mesure témoigne à la fois de l'ampleur de la pauvreté et de sa fréquence.
		\item \textbf{ Unemployment, total (\% of total labor force):} Fraction de la population active qui est sans emploi mais qui est disponible pour et à la recherche d'un emploi.
		\end{itemize}
	\subsubsection{Recherche des sources WEB des variables souhaitées}
	\par
	\begin{Huge}{ Liste des cibles (sites web et bases de données publiques) du Crawling }
		\end{Huge}
	\subsubsection{Extraction et formatage des variables souhaitées}\label{crawl}
	\par
	 \begin{Huge}{ Workflow du  Crawling }
	 		\end{Huge}
	
	\subsection{Préparation des données externes}
	Cette section s’intéresse à l'application des forêts de décisions aléatoires à des fins de sélection de variables. Le but ici est double : d'abord introduire le comportement de l'indexation de l'importance des variables en utilisant les forêts aléatoires et l'utiliser pour proposer un algorithme à deux phases pour la sélection de variables à la base de leur importance.\par
	La stratégie générale se résume en un classement des variables exogènes\footnote{à savoir les variables externes "crawlées" dans la section \ref{crawl}.} en utilisant le score d'importance de ces variables introduit par les forêts aléatoires puis une sélection ascendante itérative des variables.
	\subsubsection{Introduction aux forêts de décision aléatoires}
	Les FA\nomenclature{\textbf{FA : }}{Forêts aléatoires} est un algorithme populaire et très efficient basé appartenant aux méthodes d'agrégation pour les problèmes de régression et de classification, introduit par Breiman\cite{BREI01}, et apparaît dans les application de 'Machine Learning'\footnote{L'apprentissage automatique ou apprentissage statistique (machine learning en anglais), champ d'étude de l'intelligence artificielle, concerne la conception, l'analyse, le développement et l'implémentation de méthodes permettant à une machine (au sens large) d'évoluer par un processus systématique, et ainsi de remplir des tâches difficiles ou impossibles à remplir par des moyens algorithmiques plus classiques.} à la fin du dernier millénaire\cite{DITRI99}. Les FA deviennent de plus en plus populaires et semblent être très robustes dans beaucoup d'applications bien qu'ils ne soient pas clairement théorisés mathématiquement\cite{BIA08}. Une introduction sommaire des FA est donnée en annexe A1.
	\par

	Le principe des FA est de combiner plusieurs arbres ($ê_i$) de décision CART\cite{BREI84} en utilisant plusieurs échantillons bootstrap de l'ensemble d'apprentissage \textbf{$L_n$} et de choisir aléatoirement à chaque nœud un sous-ensemble de \textit{k} variables exogènes $X_i$. 
	\par
	L'agrégation à laquelle procèdent les FA est d’autant plus performante que la corrélation entre les prédicteurs agrégés  (arbres CART) est faible. Afin de diminuer cette corrélation, Breiman\cite{BREI01} propose de rajouter une couche d’aléa dans la construction des
	prédicteurs.  Nous attirons l'attention du lecteur vers l'annexe A2 pour une note sur CART et son utilisation au sein des FA. Sommairement, à chaque étape de CART, \textit{k} variables sont sélectionnées aléatoirement parmi les \textit{p} et la meilleure coupure est sélectionnée uniquement sur ces \textit{k} variables : \par
	\textbf{Algorithme Forêts aléatoires}
	\begin{itemize}
	\item[\textbf{Entrées:}]
	\item x, une nouvelle observation à prévoir.
	\item \textit{$L_n$}, l'échantillon.
	\item \textit{B}, le nombre d'arbres.
	\item \textit{k} $\in \mathbb{N}^* $, le nombre de variables candidates pour découper un nœud.
	\end{itemize}
	Pour i = 1,...,\textit{B}:
	\begin{itemize}
	\item Tirer un échantillon bootstrap dans \textit{$L_n$}.
	\item Construire un arbre CART sur cet échantillon bootstrap, chaque coupure est sélectionnée
	en minimisant la fonction de coût de CART sur un ensemble de \textit{k} variables choisies au
	hasard parmi les \textit{p}. On note $ê(.,\theta_i)$ l’arbre construit.
	\item[\textbf{Sortie:}]L’estimateur ${ê(x) =  \frac{1}{B} \sum_{i=1}^{B} ê_i(x,\theta_i)}$
	\end{itemize}	
	\subsubsection{Sélection des variables et élagage des données externes}
	Parmi les nombreuses sorties proposées par la fonction randomForest, deux se révèlent particulièrement intéressantes. \textbf{L'erreur Out-of-Bag} et \textbf{Le score FA d'importance des variables}. Nous définissons celles-ci rigoureusement dans l'annexe A3. Ces deux sorties nous permettent de proposer la procédure suivante à deux phases pour l'élimination itérative des variables les moins pertinentes:
	\paragraph{Phase 1. Élimination initiale et classement:}\begin{itemize}
	\item Nous modélisons par FA notre ensemble de données initial, comprenant la totalité des variables exogènes listées dans la section \ref{exoList}.
	\item Nous calculons les score FA d'importance des variables et nous éliminons les variables d'importance triviale.
	\item Nous classons les \textit{m} variables restantes par ordre descendant des score FA d'importance
	\end{itemize}
	\paragraph{Phase 2. Sélection des variables:}
	\begin{itemize}
	\item Nous construisons la collection itérative des modèles de FA employant les \textit{$\alpha$} premières variables selon leur classement de score d'Importance. Pour ${\alpha \in [1,\textit{m}]}$.
	\item Nous retenons les variables du modèle ayant réussi la plus faible erreur OOB.
	\end{itemize}
	