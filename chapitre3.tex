\chapter{Collecte et étude statistique}
\epigraph{Economists and agronomists are locked in debate about likely
future yields. Since the method of the economists is to predict
future outcomes from past performance, economists expect
success to continue. And since for the scientists future success
depends on discoveries they will have to make and do not now
know how to make, the scientists are doubtful. At its core, this is
a disagreement about the pace of technical change.}{Robert
Socolow}	
\cleardoublepage

	\section{Collecte des données}
	\subsection{Compréhension des données locales\protect\footnote{Données disponibles au sein du portail Business Intelligence de l'OCP}}
	\subsection{Extension des données aux banques externes}
	Comme rappellé dans la section \ref{CGP}. Les travaux de nos prédécesseurs dans leurs efforts de moderniser le système d'information se sont arrêtés à automatiser l'archivage des données se rapportant aux historiques de ventes en terme de prix et volumes en un premier lieu\cite{CHEMLAL} avant de concevoir le socle OLAP\footnote{le traitement analytique en ligne (OnLine Analytical Processing, OLAP) est un type d'application informatique orienté vers l'analyse sur-le-champ d'informations selon plusieurs axes, dans le but d'obtenir des rapports de synthèse} dans la vue de générer des rapports synthétiques concernant les historiques des échanges du marché des phosphates.
	\par
	L'étendue des données disponibles au sein du système d'information de l'OCP ne pouvant donner qu'une vue réduite de la situation du marché et ne peuvent de par leur définition réveler les structures socioéconomiques,  et de politiques agraires sous-jacentes aux demandes en produits phosphatés, nous nous attellerons à la tâche d'enrichir cette base de données historiques par des données quantitatives concernant les pays du monde.
	\par
	Nous décrivons dans ce qui suit ce processus. 
	\subsubsection{Recherche, énumération et extraction des données depuis des banques WEB}
	Guidés par nos lectures bibliographiques résumées dans les sections \ref{read1} et \ref{read2}, nous nous fixons l'objectif de récolleter les indicateurs de politiques agraires et socioéconomiques suivants:
	\small{
	\begin{description}[]
	\item[Access to electricity, rural (\% of rural population):] Fraction de paysans ayant accès à l'électricité.
	\item[Access to non-solid fuel, rural (\% of rural population):] Fraction de paysans ayant accès aux fuels non solides.
	\item[Account at a financial institution (\% age 15+):] Fraction de personnes agées de +15 ans ayant un compte chez une institution banquaire.
	\item[Net enrollment rate, primary (\% of primary school age children):] Fraction d'enfants scolarisés.
	\item[Net national income per capita (constant 2005 US\$):] PIB\footnote{Produit intérieur brut, un des agrégats majeurs des comptes nationaux, il vise à quantifier — pour un pays et une année donnés — la valeur totale de la « production de richesse » effectuée par les agents économiques résidant à l’intérieur de ce territoire (ménages, entreprises, administrations publiques).} par habitant, inflation ajustée au dollar US fin 2005.
	\item[Literacy rate, adult total (\% of people ages 15 and above):] Fraction de personnes alphabétisées agées de +15ans.
	\item[Agricultural irrigated land (\% of total agricultural land):] Fraction de terres irriguées parmi les terres exploitées pour l'agriculture.
	\item[Agricultural land (\% of land area):] Fraction des terres agricoles de la surface totale du pays.
	\item[Agricultural tractors per 100 sq. km of arable land:] Nombre de tracteurs par 100 km² de terres arables.
	\item[Agriculture, value added (\% of GDP):] Fraction de la valeur ajoutée agricole du PIB.
	\item[Agriculture value added per worker (constant 2005 US\$):] Valeur ajoutée agricole par ouvrier agricol, inflation ajustée au dollar US fin 2005.
	\item[All education staff compensation, total (\% of total expenditure in public institutions):] Fraction des dépenses en éducation des depenses en institutions publiques.
	\item[Annual freshwater withdrawals, agriculture (\% of total freshwater withdrawal):] Fraction du volume d'eau utilsée à des fins agricoles de la totalité de l'eau consomée.
	\item[Arable land (\% of land area):] Fraction de terres arables de la surface totale du pays.
	\item[Arable land (hectares per person):] Nombre d'hectares de terres arables par personne.
	\item[Birth rate, crude (per 1,000 people):] Nombre de naissances par 1000 personnes.
	\item[Cereal yield (kg per hectare):] Rendement des céréales en kilogramme par hectare.
	\item[Commercial bank branches (per 100,000 adults):] Nombre d'agences banquaires par 100,000 adultes.
	\item[Consumer price index (2010 = 100):] L'indice des prix à la consommation (IPC) mesure l'évolution du niveau moyen des prix des biens et services consommés par les ménages, pondérés par leur part dans la consommation moyenne des ménages. Harmonisé pour permettre une comparaison entre les pays à fin 2010.
	\item[Cost to export (US\$ per container):] Coût en Dollars US de l'export d'un conteneur de marchandises.
	\item[Cost to import (US\$ per container):] Coût en Dollars US de l'import d'un conteneur de marchandises.
	\item[Crop production index (2004-2006 = 100):]
	\item[Depth of the food deficit (kilocalories per person per day):]
	\item[Diabetes prevalence (\% of population ages 20 to 79):]
	\item[Droughts, floods, extreme temperatures (\% of population, average 1990-2009):]
	\item[Electric power consumption (kWh per capita):]
	\item[Employment in agriculture (\% of total employment):]
	\item[Energy use (kg of oil equivalent per capita):]
	\item[Fertilizer consumption (\% of fertilizer production):]
	\item[Fertilizer consumption (kilograms per hectare of arable land):]
	\item[Food exports (\% of merchandise exports):]
	\item[Food imports (\% of merchandise imports):]
	\item[Food production index (2004-2006 = 100):]
	\item[Forest area (\% of land area):]
	\item[Forest area (sq. km):]
	\item[GDP per capita (constant 2005 US\$):]
	\item[GDP (constant 2005 US\$):]
	\item[Household final consumption expenditure (constant 2005 US\$):]
	\item[International tourism, number of arrivals:]
	\item[Internet users (per 100 people):]
	\item[Investment in transport with private participation (current US\$):]
	\item[Investment in water and sanitation with private participation (current US\$):]
	\item[Labor force with primary education (\% of total):]
	\item[Labor force with secondary education (\% of total):]
	\item[Labor force with tertiary education (\% of total):]
	\item[Labor tax and contributions (\% of commercial profits):]
	\item[Land area (sq. km):]
	\item[Land under cereal production (hectares):]
	\item[Lending interest rate (\%):]
	\item[Life expectancy at birth, total (years):]
	\item[Livestock production index (2004-2006 = 100):]
	\item[Logistics performance index: Overall (1=low to 5=high):]
	\item[Long-term unemployment (\% of total unemployment):]
	\item[Low-birthweight babies (\% of births):]
	\item[Market capitalization of listed domestic companies (current US\$):]
	\item[Mobile cellular subscriptions (per 100 people):]
	\item[Net migration:]
	\item[New businesses registered (number):]
	\item[People practicing open defecation, rural (\% of rural population):]
	\item[Permanent cropland (\% of land area):]
	\item[Population density (people per sq. km of land area):]
	\item[Population growth (annual \%):]
	\item[Population, total:]
	\item[Prevalence of overweight, weight for height (\% of children under 5):]
	\item[Prevalence of severe wasting, weight for height (\% of children under 5):]
	\item[Prevalence of stunting, height for age (\% of children under 5):]
	\item[Prevalence of undernourishment (\% of population):]
	\item[Prevalence of underweight, weight for age (\% of children under 5):]
	\item[Public credit registry coverage (\% of adults):]
	\item[Risk premium on lending (lending rate minus treasury bill rate, \%):]
	\item[Rural land area (sq. km):]
	\item[Rural population:]
	\item[Rural population (\% of total population):]
	\item[Rural population growth (annual \%):]
	\item[Rural poverty gap at national poverty lines (\%):]
	\item[Rural poverty headcount ratio at national poverty lines (\% of rural population):]
	\item[Scientific and technical journal articles:]
	\item[Short-term debt (\% of total reserves):]
	\item[Surface area (sq. km):]
	\item[Total reserves (includes gold, current US\$):]
	\item[Total tax rate (\% of commercial profits):]
	\item[Trade (\% of GDP):]
	\item[Unemployment, total (\% of total labor force):]	
	\end{description}
	}
	\subsubsection{Classement des variables par ordre d'importance et sélection}
	\section{Audit de la qualité des données recolletées}
	\subsection{Données locales}
	\subsection{Données externes}
	\section{Analyse causale}
	\section{Analyse en séries chronologiques et projections}