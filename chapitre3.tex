\chapter{Collecte et étude statistique}
\epigraph{Economists and agronomists are locked in debate about likely
future yields. Since the method of the economists is to predict
future outcomes from past performance, economists expect
success to continue. And since for the scientists future success
depends on discoveries they will have to make and do not now
know how to make, the scientists are doubtful. At its core, this is
a disagreement about the pace of technical change.}{Robert
Socolow}	
\cleardoublepage

	\section{Collecte des données}
	\subsection{Compréhension des données locales\protect\footnote{Données disponibles au sein du portail Business Intelligence de l'OCP}}
	\subsection{Extension des données aux banques externes}
	Comme rappellé dans la section \ref{CGP}. Les travaux de nos prédécesseurs dans leurs efforts de moderniser le système d'information se sont arrêtés à automatiser l'archivage des données se rapportant aux historiques de ventes en terme de prix et volumes en un premier lieu\cite{CHEMLAL} avant de concevoir le socle OLAP\footnote{le traitement analytique en ligne (OnLine Analytical Processing, OLAP) est un type d'application informatique orienté vers l'analyse sur-le-champ d'informations selon plusieurs axes, dans le but d'obtenir des rapports de synthèse} dans la vue de générer des rapports synthétiques concernant les historiques des échanges du marché des phosphates.
	\par
	L'étendue des données disponibles au sein du système d'information de l'OCP ne pouvant donner qu'une vue réduite de la situation du marché et ne peuvent de par leur définition réveler les structures socioéconomiques,  et de politiques agraires sous-jacentes aux demandes en produits phosphatés, nous nous attellerons à la tâche d'enrichir cette base de données historiques par des données quantitatives concernant les pays du monde.
	\par
	Nous décrivons dans ce qui suit ce processus. 
	\subsubsection{Recherche, énumération et extraction des données depuis des banques WEB}
	\paragraph{Recherche et énumération \newline }
	Guidés par nos lectures bibliographiques résumées dans les sections \ref{read1} et \ref{read2}, nous nous fixons l'objectif de récolleter les indicateurs de politiques agraires et socioéconomiques suivants:
	\begin{itemize}
	\item Access to electricity, rural (\% of rural population): Fraction de paysans ayant accès à l'électricité.
	\item Access to non-solid fuel, rural (\% of rural population): Fraction de paysans ayant accès aux fuels non solides.
	\item Account at a financial institution (\% age 15+): Fraction de personnes agées de +15 ans ayant un compte chez une institution banquaire.
	\item Net enrollment rate, primary (\% of primary school age children): Fraction d'enfants scolarisés.
	\item Net national income per capita (constant 2005 US\$): PIB\footnote{Produit intérieur brut, un des agrégats majeurs des comptes nationaux, il vise à quantifier — pour un pays et une année donnés — la valeur totale de la « production de richesse » effectuée par les agents économiques résidant à l’intérieur de ce territoire (ménages, entreprises, administrations publiques).} par habitant, inflation ajustée au dollar US fin 2005.
	\item Literacy rate, adult total (\% of people ages 15 and above): Fraction de personnes alphabétisées agées de +15ans.
	\item Agricultural irrigated land (\% of total agricultural land): Fraction de terres irriguées parmi les terres exploitées pour l'agriculture.
	\item Agricultural land (\% of land area): Fraction des terres agricoles de la surface totale du pays.
	\item Agricultural tractors per 100 sq. km of arable land: Nombre de tracteurs par 100 km² de terres arables.
	\item Agriculture, value added (\% of GDP): Fraction de la valeur ajoutée agricole du PIB.
	\item Agriculture value added per worker (constant 2005 US\$): Valeur ajoutée agricole par ouvrier agricol, inflation ajustée au dollar US fin 2005.
	\item All education staff compensation, total (\% of total expenditure in public institutions): Fraction des dépenses en éducation des depenses en institutions publiques.
	\item Annual freshwater withdrawals, agriculture (\% of total freshwater withdrawal): Fraction du volume d'eau utilsée à des fins agricoles de la totalité de l'eau consomée.
	\item Arable land (\% of land area): Fraction de terres arables de la surface totale du pays.
	\item Arable land (hectares per person): Nombre d'hectares de terres arables par personne.
	\item Birth rate, crude (per 1,000 people): Nombre de naissances par 1000 personnes.
	\item Cereal yield (kg per hectare): Rendement des céréales en kilogramme par hectare.
	\item Commercial bank branches (per 100,000 adults): Nombre d'agences banquaires par 100,000 adultes.
	\item Consumer price index (2010 = 100): L'indice des prix à la consommation (IPC) mesure l'évolution du niveau moyen des prix des biens et services consommés par les ménages, pondérés par leur part dans la consommation moyenne des ménages. Harmonisé pour permettre une comparaison entre les pays à fin 2010.
	\item Cost to export (US\$ per container): Coût en Dollars US de l'export d'un conteneur de marchandises.
	\item Cost to import (US\$ per container): Coût en Dollars US de l'import d'un conteneur de marchandises.
	\item Crop production index (2004-2006 = 100): 
	L'indice de production des cultures montre la production agricole pour chaque année par rapport à la période de base de 2004 à 2006. Cet indice porte sur l'ensemble des cultures à l'exception des cultures fourragères. Les regroupements par région et par revenu des indices de production de la FAO sont calculés à partir des valeurs sous-jacentes en dollars US et normalisés par rapport à la période de référence de 2004 à 2006.
	\item Droughts, floods, extreme temperatures (\% of population, average 1990-2009): Pourcentage moyen annuel entre 1990 et 2009 de la population affectée par les catastrophes naturelles classifiées comme sécheresses, inondations et évènements climatiques extrêmes.
	\item Employment in agriculture (\% of total employment): Fraction des ouvriers agricoles de l'ensemble des employés.
	\item Food production index (2004-2006 = 100): L'indice de production alimentaire porte sur les cultures vivrières qui sont considérées comme comestibles et qui contiennent des nutriments et normalisées par rapport à la période de référence de 2004 à 2006.
	\item GDP per capita (constant 2005 US\$): PIB par habitant. Inflation ajustée à fin 2005.
	\item Household final consumption expenditure (constant 2005 US\$):  La consommation privée désigne la valeur marchande de tous les biens et services, y compris les produits durables achetés par les ménages.
	\item Lending interest rate (\%): Le taux d'intérêt perçu par les banques sur les prêts accordés aux clients.	
	\item Life expectancy at birth, total (years): L'espérance de vie à la naissance indique le nombre d'années qu'un nouveau-né devrait vivre si les règles générales de mortalité au moment de sa naissance devaient rester les mêmes tout au long de sa vie.
	\item Livestock production index (2004-2006 = 100): L'indice de production animale comprend la production de viande et de lait de toutes sources, les produits laitiers tels que le fromage, les œufs, le miel, la soie brute, la laine ainsi que les peaux et les cuirs.
	\item Logistics performance index: Overall (1=low to 5=high): La note globale de l'indice de performance de la logistique reflète les perceptions relatives à la logistique d'un pays basées sur l'efficacité des processus de dédouanement, la qualité des infrastructures commerciales et des infrastructures de transports connexes, la facilité de l'organisation des expéditions à des prix concurrentiels, la qualité des services d'infrastructure, la capacité de suivi et de traçabilité des consignations et la fréquence avec laquelle les expéditions arrivent au destinataire dans les délais prévus. L'indice varie continuellement de 1 à 5 et la note la plus élevée représente la meilleure performance.
	\item Low-birthweight babies (\% of births): Fraction des nouveau-nés pesant moins de 2 500 grammes des naissances totales.
	\item Net migration: Nombre d'immigrants total moins le nombre d'émigrants annuel, comprenant à la fois les citoyens et les non citoyens.
	\item Permanent cropland (\% of land area): Fraction des terres occupées par des cultures pour de longues périodes et qui doivent être replantées après chaque récolte de la surface totale du pays.
	\item Population density (people per sq. km of land area): Densité des habitants en personne par km².
	\item Population growth (annual \%): Croissance relative annuelle de la population.
	\item Rural population (\% of total population): Fraction rurale de la population.
	\item Rural poverty gap at national poverty lines (\%): L'écart de pauvreté par rapport au seuil national de la pauvreté en milieu rural est le manque à gagner pour remonter au-dessus du seuil de la pauvreté (en considérant que les non pauvres ont un manque à gagner de zéro) exprimé en pourcentage du seuil national de la pauvreté en milieu urbain.Cette mesure témoigne à la fois de l'ampleur de la pauvreté et de sa fréquence.
	\item Unemployment, total (\% of total labor force): Fraction de la population active qui est sans emploi mais qui est disponible pour et à la recherche d'un emploi.
	\end{itemize}
	\subsubsection{Classement des variables par ordre d'importance et sélection}
	\section{Audit de la qualité des données recolletées}
	\subsection{Données locales}
	\subsection{Données externes}
	\section{Analyse causale}
	\section{Analyse en séries chronologiques et projections}