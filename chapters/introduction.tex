\addcontentsline{toc}{chapter}{Introduction}
\chapter*{Introduction}
\epigraph{Agriculture is not crop production as popular belief holds - it's the production of food and fiber from the world's land and waters. Without agriculture it is not possible to have a city, stock market, banks, university, church or army. Agriculture is the foundation of civilization and any stable economy.}{Allan Savory}

\paragraph{}
Presque toutes les décisions prises par un gestionnaire ont besoin d'une prévision. S'il a une idée de ce qui se passera dans l'avenir, celui-ci peut prendre des décisions de gestion appropriées. Il a également besoin d'évaluer l'effet de ses décisions actuelles sur l'avenir afin que les bonnes décisions soient prises aujourd'hui pour créer une condition souhaitée demain.
\paragraph{}
Pour une organisation commercialisant et produisant les engrais, les estimations de la demande et des parts de marché sont indispensables pour les décisions stratégiques et celles concernant l'allocation des ressources. Les pays en développement, confrontés à des problèmes de faible productivité agricole, la croissance démographique et les besoins alimentaires, reconnaissent le rôle crucial des engrais au sein de leur politique de sécurité alimentaire. Les prévisions de la demande d'engrais à court terme et de la consommation potentielle à long terme sont essentielles pour la détermination des politiques appropriées en matière de production alimentaire et l'utilisation des engrais. De même, le mouvement mondial des engrais, leurs tendances des prix et des investissements dans de nouvelles installations de production sont influencés par les attentes de la demande.
\paragraph{}
La puissance de calcul mondiale augmentant de façon exponentielle, notre capacité à rassembler, stocker et analyser les données augmente également de façon spectaculaire. Les données sont plus abondantes dans presque toutes les industries et applications académiques. L'industrie agricole ne fait pas l'exception. Dans un certain sens, il y a plus de données au sein de l'industrie agricole que dans la plupart des autres.\cite{MIT-BIGDATA}.
\paragraph{}
Les informations et les connaissances acquises par les entreprises de la technologie agricoles sont généralement brevetées, surveillées, et utilisées à des fins de concurrence dans le marché. Certains pensent que de meilleures décisions en matière de chaîne d'exploitation et d'approvisionnement pourraient être établies par la compilation minutieuse et l'analyse de segments particuliers de ces données. L'OCP\footnote{Office Chérifien des Phosphates} rejoint ce sentiment.
\paragraph{}
 Dans le cadre de notre projet de fin d'études la mission - en un premier lieu - de mettre en place un mécanisme d'extraction, de transformation et de chargements de données non structurées. La nature non-structurée de celles-ci interdit tout approche BI classique en matière d'intégration des données et a nécessité de notre part un développement spécifique d'un moteur de Parsing qui automatisera les tâche de structuration et d'extraction. En un second lieu, notre encadrement à l'OCP nous as confié l'accès à l'ensemble des données en la possession du département hôte de notre stage avec la charge de réunir des éléments d'informations décisifs à la concurrence du groupe OCP. Nous nous sommes fixés le but de construire des modèles statistiques de prévision quant à la demande des produits dérivés phosphatés. La perspective finale est d'intégrer ces modèles au sein d'un module logiciel de prévisions dispensant des projections de marché à la demande, réalisant ainsi un moteur de prévision.
\paragraph{}
Ce mémoire présente les différentes étapes de réalisation de notre projet. Il se compose de quatre
chapitres dont la description est comme suit.
Le premier chapitre aborde le contexte général du projet. Il présente l’organisme d’accueil, la
direction commerciale et dévoile la motivation et les objectifs du projet. Ensuite, on passe à la
démarche suivie avant de terminer avec le planning.
Le deuxième chapitre est dédié à l’analyse et spécifications des besoins. Ce qui se traduit par
une étude de l’existant, ses limites et le recensement des différents besoins de la direction
commerciale, pour aboutir à un cadrage fonctionnel et technique de la solution proposée. Nous procédons lors du troisième chapitre à concevoir le moteur d'extraction (notre ETL développé par nous mêmes), testons sa robustesse à travers un audit de la qualité des données qu'il transforme. Le chapitre 3 sera également pour nous l'occasion de nous placer dans la perspective de notre problématique et enrichir les données disponibles à l'OCP pour permettre une analyse causale de la demande en fertilisants dont nous construisant le modèle au chapitre quatre. Pour clore ce mémoire, nous résumons les résultats et les acquis réalisés lors de notre stage, les
difficultés rencontrées et les perspectives de ce travail.