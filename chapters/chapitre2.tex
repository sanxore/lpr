\chapter{Analyse et spécification}
\epigraph{If a student takes the whole series of my folklore courses including the graduate seminars, he or she should learn something about fieldwork, something about bibliography, something about how to carry out library research, and something about how to publish that research.}{Alan Dundes}

\cleardoublepage

\section{Analyse de l’existant}
\section{Revue de littérature}
\subsection{Qu'est ce que le Data Mining ?}
La fouille de données, français pour le "Data Mining" est le processus de recherche et de découverte d'auparavant inconnus et potentiellement intéressants modèles dans les grands ensembles de données \cite{def-DM}. L'information 'minée' est typiquement représentée par un modèle de la structure sémantique de l'ensemble de données, où le modèle peut être utilisé sur de nouvelles données à des fins de prédiction ou de classification. Alternativement, des experts humains du métier en question peuvent choisir d'examiner manuellement le modèle, à la recherche d'éléments qui expliqueraient des caractéristiques précédemment mal comprises ou inconnues du domaine d'étude.
\paragraph{}
Brynjolfsson, Hitt, et Kim\cite{data-driven-des} ont mené des recherches empiriques et ont conclu que la performance des organisations est directement liée à leur capacité à prendre des décisions dirigées par les données. Il est donc important de comprendre les facteurs de succès nécessaires pour adopter des techniques de fouilles de données dans les organisations.
\subsection{Les études de prévisions en matière de fertilisants}\label{read1}
\subsection{Prévision à court-terme de la demande de fertilisants}\label{read2}
\section{Compréhension du problème}
\section{Spécification}
	\subsection{Spécification fonctionelle}
		\begin{figure}[H]
				\centering
				\fbox{\includegraphics[width=.7\textwidth]{ch2-images/archi}}
				
				\label{fig:archi}
			\caption{Architecture Globale.}
		\end{figure}
	\subsection{Spécification technique}
	\subsubsection{Outils de réalisation}
		\begin{itemize}
			\item
			\textbf{Le langage de programmation Python : } 
				\begin{itemize}
					\item[$\textasteriskcentered$] Python est un langage de programmation objet, multi-paradigme et multiplate-forme. Il favorise la programmation impérative structurée, fonctionnelle et orientée objet.
					\item[$\textasteriskcentered$] Il est conçu pour optimiser la productivité des programmeurs en offrant des outils de haut niveau et une syntaxe simple à utiliser.
					\item[$\textasteriskcentered$] \textbf{Pourquoi Python dans notre projet ?}
					\begin{itemize}
						\item[\textbf{+}] Calculatrice vectorielle évoluée.
						\item[\textbf{+}] Traitement de fichier texte.
						\item[\textbf{+}] Scripts, ou commandes Unix pour traitements de fichiers par lots.
						\item[\textbf{+}] Langage "Glue" pour enchaîner les traitements par différents programmes.\\\underline{Cas d'utilisation dans notre projet:} Parsing PDF et enregistrement sur fichiers, repris par un code pour structuration des données et traitement.
					\end{itemize}
				\end{itemize}
			\item
			\textbf{Le langage statistique R : } 
				\begin{itemize}
					\item[$\textasteriskcentered$] R est un environnement permettant de faire des analyses statistiques et de produire des graphiques évolués.
					\item[$\textasteriskcentered$] C’est également un langage de programmation complet et mature. Sa licence est open-source, son utilisation est gratuite, même dans le contexte de l’entreprise ou de la formation.
					\item[$\textasteriskcentered$] L’environnement R intègre de nombreuses fonctionnalités pour l'acquisition, le nettoyage et la modélisation de données.
					\item[$\textasteriskcentered$] \textbf{Pourquoi R dans notre projet ?}
					\begin{itemize}
						\item[\textbf{+}] C’est un outil très puissant et très complet, particulièrement bien adapté pour la mise en œuvre informatique de méthodes statistiques. Il est plus difficile d’accès que certains autres logiciels du marché (comme SPSS ou Matlab par exemple), car il n’est pas conçu pour être utilisé à l’aide de «clics» de souris dans des menus. Mais son approche par l'écriture de code informatique pour l'analyse statistique lui confère la flexibilité désiré pour un projet de Data Science.
						\item[\textbf{+}] L’approche est pédagogique puisqu’il faut maîtriser les méthodes statistiques pour parvenir à les mettre en œuvre.
						\item[\textbf{+}] l’outil est très efficace lorsque l’on domine le langage puisque l’on devient alors capable de créer ses propres outils, ce qui permet ainsi d’opérer des analyses très sophistiquées sur les données en retenant tous les avantages de la programmation modulaire dont la ré-utilisabilité et la généricité du code produit.
					\end{itemize}
				\end{itemize}
			\item
			\textbf{Le langage de scripting système BASH : } 
				\begin{itemize}
					\item[$\textasteriskcentered$] Un script BASH est une suite d’instructions, de commandes qui constituent un scénario d'actions. C’est un fichier texte que l’on peut exécuter, c’est à dire, lancer comme une commande.
					\item[$\textasteriskcentered$] \textbf{Pourquoi BASH dans notre projet ?}
					\begin{itemize}
						\item[\textbf{+}] le shell est l'interface de tous les jours en UNIX. Bien connaître son shell permet d'économiser beaucoup d'efforts.
						\item[\textbf{+}] le shell est universel: peu importe le système UNIX, les tâches d'automatisation des appels systèmes privilégient les scripts BASH.
						\item[\textbf{+}] Il est plus aisé de programmer en BASH, par rapport par exemple à C; le Shell n'a pas été conçu pour être minimal ou théoriquement élégant; il a été conçu pour être flexible et pratique. Ainsi, dans bien des cas on va s'en servir pour automatiser les tâches routinières du système.\\
						\underline{Cas d'utilisation dans notre projet:} Automatisation de la création des arborescences de structuration des données à consolider ainsi que les appels systèmes de leur labellisation.
					\end{itemize}
				\end{itemize}
		\end{itemize}
		
		\begin{figure}[H]
			\begin{subfigure}[b]{.3\textwidth}
				\centering
				\includegraphics[width=\textwidth]{ch2-images/python}
				\caption{Logo Python}
				\label{fig:python}
			\end{subfigure}
			\begin{subfigure}[b]{.3\textwidth}
					\centering
					\includegraphics[width=.8\textwidth]{ch2-images/R}
					\caption{Logo R}
					\label{fig:R}
			\end{subfigure}
			\begin{subfigure}[b]{.3\textwidth}
					\centering
					\includegraphics[width=.8\textwidth]{ch2-images/Bash}
					\caption{Logo Bash}
					\label{fig:bash}
			\end{subfigure}
			\caption{Logos des outils utilisés}
		\end{figure}
