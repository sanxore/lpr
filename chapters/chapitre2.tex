\chapter{Analyse et spécification}
\epigraph{If a student takes the whole series of my folklore courses including the graduate seminars, he or she should learn something about fieldwork, something about bibliography, something about how to carry out library research, and something about how to publish that research.}{Alan Dundes}

\cleardoublepage

\section{Analyse de l’existant}
\section{Revue de littérature}
\subsection{Qu'est ce que le Data Mining ?}
La fouille de données, français pour le "Data Mining" est le processus de recherche et de découverte d'auparavant inconnus et potentiellement intéressants modèles dans les grands ensembles de données \cite{def-DM}. L'information 'minée' est typiquement représentée par un modèle de la structure sémantique de l'ensemble de données, où le modèle peut être utilisé sur de nouvelles données à des fins de prédiction ou de classification. Alternativement, des experts humains du métier en question peuvent choisir d'examiner manuellement le modèle, à la recherche d'éléments qui expliqueraient des caractéristiques précédemment mal comprises ou inconnues du domaine d'étude.
\paragraph{}
Brynjolfsson, Hitt, et Kim\cite{data-driven-des} ont mené des recherches empiriques et ont conclu que la performance des organisations est directement liée à leur capacité à prendre des décisions dirigées par les données. Il est donc important de comprendre les facteurs de succès nécessaires pour adopter des techniques de fouilles de données dans les organisations.
\subsection{Les études de prévisions en matière de fertilisants}\label{read1}
\subsection{Prévision à court-terme de la demande de fertilisants}\label{read2}
\section{Compréhension du problème}
\section{Spécification}
	\subsection{Spécification fonctionelle}
	\subsection{Spécification technique}
	\subsubsection{Outils de réalisation}
		\begin{itemize}
			\item
			\textbf{Python : } 
				\begin{itemize}
					\item[$\textasteriskcentered$] Python est un langage de programmation objet, multi-paradigme et multiplate-forme. Il favorise la programmation impérative structurée, fonctionnelle et orientée objet.
					\item[$\textasteriskcentered$] Il est conçu pour optimiser la productivité des programmeurs en offrant des outils de haut niveau et une syntaxe simple à utiliser.
					\newpage
					\item[$\textasteriskcentered$] \textbf{Pourquoi Python dans notre projet ?}
					\begin{itemize}
						\item[\textbf{+}] Calculatrice vectorielle évoluée.
						\item[\textbf{+}] Traitement de fichier texte.
						\item[\textbf{+}] Scripts, ou commandes Unix pour traitements de fichiers par lots.
						\item[\textbf{+}] Langage "Glue" pour enchaîner les traitements par différents programmes. Exemple : Parsing PDF et enregistrement sur fichiers, repris par un code pour structuration des données et traitement.
					\end{itemize}
				\end{itemize}	
		\end{itemize}