\chapter{Contexte général du projet}

Ce chapitre représente une mise en contexte du projet en se focalisant dans la première partie sur son cadre à travers la présentation de la problèmatique qu'il doit résoudre. La deuxième partie aura comme but d'éclaircir l'objectif général à atteindre ainsi les différentes exigences à respecter avant d'attaquer dans la dernière partie, l'organisation méthodologique mise en œuvre pour réaliser ce projet\cite{goossens93}.
\cleardoublepage

\section{Présentation de l’OCP}

	
\section{Présentation du projet}
	\subsection{Cadre général du projet}
	\subsection{Motivation et Problématique}


\section{Planification du projet}
	\subsection{Les étapes CRISP-DM d'un projet Data Mining}
	\subsection{Le planning du projet}