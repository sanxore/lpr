\addcontentsline{toc}{chapter}{Introduction}
\chapter*{Introduction}
\epigraph{Agriculture is not crop production as popular belief holds - it's the production of food and fiber from the world's land and waters. Without agriculture it is not possible to have a city, stock market, banks, university, church or army. Agriculture is the foundation of civilization and any stable economy.}{Allan Savory}

\paragraph{}
Presque toutes les décisions prises par un gestionnaire ont besoin d'une prévision. S'il a une idée de ce qui se passera dans l'avenir, celui-ci peut prendre des décisions de gestion appropriées. Il a également besoin d'évaluer l'effet de ses décisions actuelles sur l'avenir afin que les bonnes décisions soient prises aujourd'hui pour créer une condition souhaitée demain.
\paragraph{}
L’industrie des fertilisants est à forte sollicitation de capital et est, par conséquent, sensible aux variations des coûts. Si nous savons quels types d'engrais sont susceptibles d'être exigés, où et quand, nous pouvons améliorer la qualité des décisions relatives à la production, l'approvisionnement, le placement et la promotion. Par conséquent, nous pouvons minimiser les fonds immobilisés dans les stocks, réduire les coûts d'intérêt, maximiser les réserves en devises étrangères, éviter de manquer de stock et, en général, augmenter les ventes et améliorer les profits.
\paragraph{}
Pour une organisation commercialisant et produisant les engrais, les estimations de la demande et des parts de marché sont indispensables pour les décisions stratégiques et celles concernant l'allocation des ressources. Les pays en développement, confrontés à des problèmes de faible productivité agricole, la croissance démographique et les besoins alimentaires, reconnaissent le rôle crucial des engrais au sein de leur politique de sécurité alimentaire. Les prévisions de la demande d'engrais à court terme et de la consommation potentielle à long terme sont essentielles pour la détermination des politiques appropriées en matière de production alimentaire et l'utilisation des engrais. De même, le mouvement mondial des engrais, leurs tendances des prix et des investissements dans de nouvelles installations de production sont influencés par les attentes de la demande.
\paragraph{}
La puissance de calcul mondiale augmentant de façon exponentielle, notre capacité à rassembler, stocker et analyser les données augmente également de façon spectaculaire. Les données sont plus abondantes dans presque toutes les industries et applications académiques. L'industrie agricole ne fait pas l'exception. Dans un certain sens, il y a plus de données au sein de l'industrie agricole que dans la plupart des autres. L'agriculture est l'un des métiers les plus anciens du monde. Depuis des millénaires, les pratiques agricoles ont été transmises et améliorées. Pendant des siècles, les données agricoles ont été recueillies, suivies et analysées\cite{MIT-BIGDATA}.
\paragraph{}
Les données agricoles sont variées : des rendements des cultures dans certaines zones géographiques aux indicateurs d'érosion, en passant par la météo et les conditions climatiques. Ces données sont collectées, analysées et dans la plupart des cas, partagées par une variété de sources: entreprises privées, des agences gouvernementales et des universités de recherche. Une partie de ces données sont publiées dans des revues, disponibles par le biais des bases de données sur le Web ou vendues par des prestataires d'information. Ces données varient en portée et en exhaustivité; certaines sources de données sont très granulaires tandis que d'autres sont agrégées.
Les entreprises qui produisent des produits agricoles, les équipes d'ingénieurs R\&D\nomenclature{\textbf{R\&D : }}{Recherche et Développement} et chercheurs recueillent ces données et suivent leurs tendances pour améliorer les semences, les herbicides, les pesticides, les engrais, et la technologie agricole. Les informations et les connaissances acquises par ces entreprises sont généralement brevetées, surveillées, et utilisées à des fins de concurrence dans le marché. 
\paragraph{}
Une fois ces données recueillies, elles sont utilisées par une variété de personnes et d'organisations pour améliorer l'efficacité de l'industrie agricole d’aujourd’hui. Les producteurs individuels utilisent les données des cultures, de la météo et du sol pour prendre des décisions au sujet de leur prochaine saison ainsi que pour pérenniser leurs exploitations. Les entreprises qui fournissent les semences, les produits chimiques, les engrais et d'autres nécessités agricoles surveillent ces données pour créer des prévisions de demande et des plans de production.
\paragraph{} 
Une grande partie de ces données est largement utilisée dans l'industrie. Certains pensent que de meilleures décisions en matière de chaîne d'exploitation et d'approvisionnement pourraient être établies par la compilation minutieuse et l'analyse de segments particuliers de ces données. L'OCP\footnote{Office Chérifien des Phosphates} rejoint ce sentiment.