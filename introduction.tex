\addcontentsline{toc}{chapter}{Introduction}

\chapter*{Introduction}

Le développement des nouvelles technologies de l’information et de la communication dans les domaines sanitaire et médico-social peut être considérée comme l’une des réponses aux problématiques que traverse actuellement notre système de santé.

\vspace{6pt}
\paragraphmark

La grande quantité d’informations sensibles présentes dans un système d'e-santé et le transfert de ces données entre les institutions sont des défis complexes pour les technologies protégeants la vie privée. De plus, les extensions possibles de ce système, comme une utilisation secondaire des données médicales pour la recherche, ont besoin d’une attention supplémentaire. En effet, cette divulgation de données doit utiliser une anonymisation complète des informations qui seront uniquement, par exemple, des prescriptions et des diagnostics comme cela est recommendé par les chercheurs du domaine.

\vspace{6pt}
\paragraphmark

Dans le cadre de notre projet de fin de deuxième année, nous étions invités à développer une application mobile de santé préservant la vie privée des patients en se basant sur une approche intitulée PPAMH \footnote{Privacy-Preserving Approach for Mobile Healthcare.}.

\vspace{6pt}
\paragraphmark

Ce rapport est organisé en trois chapitres :

\vspace{6pt}
\paragraphmark

\begin{itemize}
	\item 
Le premier chapitre présente le contexte général du projet en spécifiant la problèmatique à résoudre, les objectifs à atteindre, les différentes exigences à respecter ainsi la démarche suivie pour le réaliser avant de citer quelques lois relatives à la protection de la vie privée.
	\item
Les besoins de sécurité des applications e-santé mobiles seront décrites dans le deuxième chapitre tout en montrant l'importance des politiques de privacy et de sécurité.
	\item
Enfin, le dernier chapitre est réservé à la réalisation de l'application Health+.
\end{itemize}

