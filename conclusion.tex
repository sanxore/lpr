\addcontentsline{toc}{chapter}{Conclusion}

\chapter*{Conclusion}

Le fait de travailler sur les applications e-santé mobiles est toujours motivant, c’est pour cela que nous avons choisi ce projet en admettant qu'un tel domaine mérite d'être découvert.

\vspace{6pt}
\paragraphmark

Le projet tutoré est beaucoup plus complexe que les projets que nous avons déjà réalisés (projet de programmation en C, projet de compilation, projet SI...). Il permet de mettre en relation plusieurs enseignements, utilisés sous une dimension à la fois théorique et paratique.

\vspace{6pt}
\paragraphmark

Tout d’abord, ce projet nous a permis d’enrichir et de renforcer les connaissances que nous avons acquises durant les deux années, mais aussi apporter son savoir et ses compétences afin d’harmoniser l’efficacité du groupe. Ce projet nous a permis aussi d’acquérir d’autres qualités à savoir : écouter l’opinion de l’autre, savoir communiquer et argumenter afin d’opter pour les meilleurs choix, s’organiser sur les plans personnels et collectifs, gérer les imprévus, respecter des délais pour ne pas gêner son collègue et pour ne pas retarder tout le projet.

\vspace{6pt}
\paragraphmark

Concernant le projet, l'application Health+ est parfaitement fonctionnelle pour les parties principales, à savoir : l'interface d'authentification (basée sur le login et le mot de passe), le questionnaire pour déterminer le groupe d'un patient et mettre en place les règles adéquates. Pour le design, nous avons veillé à ce qu’il soit intuitif et innovant.

\vspace{6pt}
\paragraphmark

Pour conclure, ce projet nous a été très bénéfique. Il nous a permis de travailler en groupe sur un projet qui demande l’association de plusieurs concepts dans le domaine de la sécurité informatique et de mettre en œuvre plusieurs langages de programmation.