\addcontentsline{toc}{chapter}{Résumé}

\chapter*{Résumé}

Les nouvelles technologies de l’information et de la communication ont un effet important sur notre vie quotidienne aussi sur plusieurs disciplines professionnelles, notamment celui de la santé qui ne cesse de se developper à l’aide des innovations technologiques. Désormais le patient peut mesurer par exemple sa pression sanguine et la communiquer à son médecin à l’aide d’un smartphone, d’une tablette ou d’un PC. Cette communication doit \^etre sécurisée et fiabilisée afin de rendre possible la confiance de l'usager (patient, médecin, etc).

\vspace{6pt}
\paragraphmark

Dans le cadre de notre projet de fin de $2^{éme}$ année, nous avions pour mission : la conception et la réalisation d'une application mobile d'e-santé pour préserver la vie privée des patients en utilisant une approche nommée PPAMH (Privacy-Preserving Approach for Mobile Healthcare), tout en suivant un cycle de vie qui commence par analyser l’existant et définir les besoins jusqu’à la validation, en passant par les étapes de la conception et de la réalisation.

\vspace{35pt}
\paragraphmark

\textbf{Mots-clés}: mSanté, privacy, sécurité, PPAMH.